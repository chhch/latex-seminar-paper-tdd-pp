\begin{abstract}

\textit{Kontext und Motivation}: Für Hochschulabsolventen/innen eines Informatikstudiengangs ändert sich die Situation häufig grundlegend, sobald sie beginnen im industriellen Umfeld zu arbeiten. Die praktische Anwendung des theoretischen Wissens gewinnt im Beruf immer mehr an Bedeutung. Die Abschluss inhabende Person muss sich unter anderen mit Dingen, wie technischen Schulden, einer neuen Domäne oder der Arbeit in größeren Teams auseinandersetzten. Kommunikation und viel Feedback helfen Berufsunerfahrenen sich in dieser neuen Situation zurechtzufinden. Feedback und Kommunikation sind gleichzeitig auch zwei elementare Werte des Extreme Programmings (XP). Die Umsetzung dieser Werte erfolgt im XP unter anderen durch die beiden Praktiken Pair-Programming und testgetriebene Entwicklung.

% Abschluss inhabende Person; Person mit Abschluss
% Praktikum (direkt) nach Studienabschluss

%Das theoretische Wissen von Hochschulabsolventen ist oft ausgeprägter als ihre Fähigkeit das Wissen praktisch anzuwenden, da das Studium mehr auf die Theorie als auf die Praxis fokussiert ist. Für die Absolventen gewinnt die Anwendung des theoretischen Wissens häufig erst beim Berufseinstieg an Bedeutung. Zusätzlich unterscheidet sich die Situation in einem Unternehmen von der im Studium. Dies kann bei den Absolventen unter Umständen zu Unsicherheiten führen. Dem könnte das Vorgehen des Extrem Programmings entgegenwirken, dessen Werte u.a. Feedback, Kommunikation, Respekt und Courage sind.

\textit{Zentrale Fragen}: In der vorliegenden Arbeit wird untersucht, welche Potenziale Pair-Programming und testgetriebene Entwicklung, einzeln und in Kombination, für Berufseinsteiger haben.

%\textit{Ziele}: Anhand des aktuellen Stands der Wissenschaft soll eine Aussage über die Anwendung der beiden Praktiken gegeben werden.

%\textit{Wissenschaftlicher Beitrag}: Auswertung verschiedener Quellen in Hinblick auf das beschriebene Szenario. Ziehen eigener Rückschlüsse für die Beantwortung der gestellten Fragen. Beschreibung von Empfehlungen für die Lösung des Problems oder für weitere Untersuchungen. Dies gilt insbesondere bei der Betrachtung von der Kombination beider Prinzipien.

\textit{Ergebins}: Potenziale wurden vor allem in den Bereichen der Kommunikation, dem Übernehmen von Verantwortung, der Einarbeitung und der Stärkung des Selbstvertrauens identifiziert. Durch die Kombination von Pair-Programming und der testgetriebenen Entwicklung werden die meisten Herausforderungen angesprochen. Die Nutzung von nur einer der beiden Praktiken birgt ebenfalls viele Potenziale, kann in Einzelfall aber auch zu Problemen führen.

\end{abstract}
