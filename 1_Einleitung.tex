%!TEX root = bare_conf.tex

\section{Einleitung}

Oft haben Studierende während ihrer Ausbildung nur eingeschränkt Zeit praktische Erfahrungen in der Programmierung zu sammeln. Für Hausarbeiten, die nach Abschluss im Schrank/Papierkorb landen, ist die Codequalität bei der Entwicklung oft kein wichtiges Kriterium. Dies ändert sich bei dem Berufseinstieg, wenn Projekte über mehrere Jahre laufen und gepflegt werden müssen. Technische Schulden, schwer zu wartender oder ungetesteter Code können langfristig zum Scheitern eines Projekts führen. Dies sorgt unter Umständen dafür, dass sich Absolventen\footnote{Aus Gründen der besseren Lesbarkeit wird hier und im Folgenden auf die gleichzeitige Verwendung männlicher und weiblicher Sprachformen verzichtet. Sämtliche Personenbezeichnungen gelten gleichwohl für beiderlei Geschlecht.} an ihrem neuen Arbeitsplatz unter Druck gesetzt fühlen, selbst wenn sie die theoretischen Grundlagen beherrschen. Für den Arbeitgeber bedeutet die mangelnde Praxis der Absolventen, dass teure Fachkräfte bezahlt werden, die sich nicht nur in eine neue Domäne, sondern auch mit der praktischen Programmierung auseinandersetzten müssen, bevor sie für das Unternehmen wirtschaftlich einsetzbar sind. Feedback hilft Unsicherheiten bzw. Unklarheiten zu klären. Häufiges Feedback ist ein zentraler Wert von \textit{Extreme Programming} (XP). Erreicht wird dies im XP unter anderen durch die beiden Praktiken \textit{Pair-Programming} (PP) und \textit{testgetriebene Entwicklung} (TDD). Beim Pair-Programming wird die Software von zwei Entwicklern gemeinsam an einem Rechner implementiert. Die testgetriebene Entwicklung unterscheidet sich von der traditionellen Entwicklung (TLD) insofern, als dass die Tests nicht nach, sondern vor der Implementierung der Funktionalität geschrieben werden \cite{Beck2004ExtremeChange}.

In dem nachfolgenden Abschnitt \ref{sec:methodik} sind die Vorgehensweisen beschrieben, nach denen die Potenziale von der TDD und dem PP untersucht wurden. Im Anschluss sind im Abschnitt \ref{sec:Probleme} die Herausforderungen für Berufseinsteiger schematisch dargestellt. Darauf folgt im Abschnitt \ref{sec:tdd} und im Abschnitt \ref{sec:pp} die getrennte Untersuchung der Potenziale von TDD bzw. PP. Der nächste Abschnitt \ref{sec:TDD+PP} behandelt die Potenziale, wenn die Praktiken gemeinsam angewendet werden. Zum Schluss erfolgt in Abschnitt \ref{sec:ende} ein Fazit.
