%!TEX root = bare_conf.tex

\section{Herausforderungen für Berufseinsteiger} \label{sec:Probleme}

Für Absolventen, die ihr Informatik Studium beendet haben, stellen sich beim Berufseinstieg in dem Bereich der Software-Entwicklung einige Herausforderungen.

Das Umfeld ändert sich in der Regel. Der Berufseinsteiger muss das Unternehmen und die Mitarbeiter kennenlernen. Wahrscheinlich wird er sich in eine neue Domäne einarbeiten müssen, je nach dem im welchen Bereich das Unternehmen tätig ist. Anders als die Kommilitonen im Studium, werden die Mitarbeiter nicht mehr alle auf den selben Niveau wie er sein. Einige Mitarbeiter sind beispielsweise für die Datenbank zuständig, andere für das Frontend oder für den Austausch mit dem Kunden. Jeder Mitarbeiter verfügt über Expertenwissen auf seinem eigenen Gebiet.

Vielleicht nutzt das Unternehmen neuere, ältere oder ganz andere Technologien als die, die der Berufseinsteiger aus seinem Studium kennt. Sehr wahrscheinlich ist, dass er sich in mindestens einer Technologie neu oder vertiefend auseinandersetzten muss. Wenn es sich bei seinem neuen Arbeitsplatz nicht, um ein Startup-Unternehmen handelt, wird vermutlich auch die Entwicklung an Altsystemen in seinen Aufgabenreich fallen. Es ist nicht unwahrscheinlich, dass er sich mit viel fremden Code auseinandersetzten muss. Zusätzlich wird das Unternehmen in der Regel bestimmte Werkzeuge für die Entwicklung, Versionsverwaltung, Fehlerberichte und ähnliches bereitstellen, mit denen er sich eventuell erst vertraut machen muss.

Die Arbeitsweise unterscheidet sich in einigen Punkten von der im Studium. Die Projekte werden meist längere Laufzeiten haben, als dies bei den Haus- oder Seminararbeiten im Studium der Fall war. Daher ist beispielsweise während der Entwicklung, vermehrt auf die Vermeidung von technischen Schulden zu achten. Es könnte sein, dass der Absolvent seine Arbeit nicht mehr ganz so frei, wie im Studium, einteilen kann. Arbeiten müssen evtl. plötzlich für einen kritischen Bugfix unterbrochen werden. Nachdem der Fehler behoben ist, muss die vorherige Arbeit ohne lange Einarbeitung fortgesetzt werden können. Anders als im Studium, wo Kommilitonen und Professoren meistens erreichbar sind und schnell Rückmeldung geben, ist der Austausch im Beruf mit anderen Projektbeteiligten, wie Mitarbeitern oder Kunden vielleicht nicht mehr ohne weiteres möglich. Die Mitarbeiter arbeiten vielleicht an einem anderen Standort oder haben Urlaub, der Kunde ist nicht erreichbar.

Baytiyeh und Naja \cite{Baytiyeh2011ChallengesCareer} befragten 217 Ingenieure, darunter 78\% männlich und 22\% weiblich, aus den Bereichen Bauingenieurwesen (32\%), Maschinenbau (23\%), Elektrotechnik (27\%), Technische Informatik (10\%) und Wirtschaftsingenieurwesen (8\%) über Herausforderungen die sie im Übergang vom Studium in den Beruf sehen. Die Teilnehmer arbeiteten im Libanon (43\%), der Region um den Persischen Golf (33\%), Europa und Nordamerika (14\%) und Afrika (10\%). Die größten Herausforderungen fallen in die drei Bereiche Kommunikation, Verantwortung und Selbstvertrauen. Die drei Bereiche wurden aus den Antworten der Teilnehmer, wie folgt abgeleitet:
\begin{itemize}
\item Kommunikation
\begin{itemize}
  \item Zusammenarbeit mit Menschen aus einem anderen Bereich
  \item Umgang mit dem Vorgesetzten
\end{itemize}
\item Verantwortung
\begin{itemize}
  \item Verantwortung übernehmen
  \item Verantwortlich für Ergebnisse
  \item Arbeitsdruck
  \item Selbständig arbeiten
\end{itemize}
\item Selbstvertrauen
\begin{itemize}
  \item Angst Fehler zu machen
  \item Nicht genug zu Wissen
  \item Selbständig lernen
\end{itemize}
\end{itemize}
Die Liste wurde mit den oben beschriebenen Herausforderungen kombiniert und dient in den nachfolgenden Abschnitten als Schema, um die Potenziale von Pair-Programming und der testgetriebenen Entwicklung zu bewerten.
\begin{itemize}
\item Kommunikation
\begin{itemize}
  \item Zusammenarbeit mit Menschen aus einem anderen Bereich, kein Expertenwissen, unterschiedliches Niveau
  \item Kommunikation mit Kunden nur eingeschränkt möglich
  \item Kommunikation mit Mitarbeitern nur eingeschränkt möglich
\end{itemize}
\item Verantwortung
\begin{itemize}
  \item Verantwortung übernehmen, längere Projektlaufzeiten
  \item Verantwortlich für Ergebnisse, technische Schulden
  \item Arbeitsdruck, Unterbrechungen
  \item Selbständig arbeiten
\end{itemize}
\item Selbständig lernen, Einarbeitung
\begin{itemize}
  \item Neue Domäne
  \item Neue Technologie
  \item Altsysteme / fremder Code
  \item Neue Werkzeuge
\end{itemize}
\item Selbstvertrauen
\begin{itemize}
  \item Angst Fehler zu machen
  \item Nicht genug zu Wissen
\end{itemize}
\end{itemize}

%umfeld
%\item Neue (fremde) Domäne, in die sich der Entwickler einarbeiten muss
%\item Nicht mehr alle auf dem gleichen Niveau, stärke Unterscheidung zwischen den oder verschiedene Fähigkeiten (Frontendexperte, Datenbankexperte...)
%\item Neues Umfeld, Unternehmensphilosophie, Mitarbeiter

%technisch
%\item Nicht auf den neusten Stand der Technik oder auf einem für das Unternehmen „zu“ neuem Stand. Andere Technologien. Viele neue Technologien.
%\item Altsysteme / viel fremder Code
%\item Vorgaben bestimmter (unbekannter) Werkzeuge für Versionsverwaltung, Entwicklung, Continous Delivery...

%arbeitsweise
%\item Lange Projektlaufzeiten. Technische Schulden.
%\item Weniger Freiheiten bei der Zeiteinteilung. Arbeit unterbrechen für ein Bugfix an einem anderen Projekt.
%\item Größe Projektteams möglich, evtl. verteilt in verschiedenen Städten/Ländern, (tatsächlich) unterschiedliche Rollen
%\item Kundenkontakt. Kunde nicht immer erreichbar.

%persönlich
%\item Mehr Verantwortung. Verantwortlich für die Ergebnisse. Unter Druck arbeiten. Selbständig Arbeiten \cite{Baytiyeh2011ChallengesCareer}
%\item Angest Fehler zu machen. Nicht genug zu wissen. Selbstänidg zu lernen \cite{Baytiyeh2011ChallengesCareer}
