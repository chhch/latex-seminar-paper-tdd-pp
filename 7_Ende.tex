\section{Fazit} \label{sec:ende}

Die Auswertungen der Studien ergab, dass hinsichtlich der Produktivität, internen Codequalität und externen Codequalität kaum Unterschiede zwischen der testgetriebenen und der traditionellen Entwicklung bestehen bzw. nicht eindeutig ist, welches Verfahren vorzuzuziehen ist. Hier sind weitere Studien mit mehr Teilnehmern, längeren Laufzeiten und genauerer Beobachtung, ob TDD korrekt angewendet wurde, abzuwarten. Dennoch kann die TDD den Berufseinsteigern bei Herausforderungen in den Bereichen Kommunikation, Verantwortung, Einarbeitung und Selbstvertrauen helfen. Voraussetzung dafür ist jedoch, dass der Berufseinsteiger Kenntnisse von der TDD hat oder motiviert ist diese zu erlangen. Sonst kann das Erlernen bzw. Anwenden der TDD schnell frustrierend werden.

Die betrachteten Studien wiesen nach, dass die Anwendung von PP hilft Programmfehler früher zu finden sowie das Codeverständnis das Design und die Codequalität zu verbessern. Studenten die PP anwenden sind seltener frustriert, haben mehr Selbstvertrauen und fühlen sich weniger isoliert. PP hilft den Berufseinsteigern mit den Herausforderungen in den Bereichen Kommunikation, Verantwortung, Einarbeitung und Selbstvertrauen umzugehen. Zusätzlich unterstützt PP den Berufseinsteiger in das Team zu integrieren. Neben Organisatorischen Schwierigkeiten, kann PP zu Problemen führen, wenn die beiden Entwickler im zwischenmenschlichen Bereich nicht harmonieren. Das könnte zum Beispiel der Fall sein, wenn der Berufseinsteiger wenig kommunikativ ist oder der Experte, mit mehrjähriger Erfahrung, sich von dem Berufseinsteiger gestört fühlt.

Beide Praktiken unterstützen den Berufseinsteiger mit seinen Mitarbeitern zu kommunizieren, mit den neuen Verantwortungen zurechtzukommen, bei der Einarbeitung bzw. mit dem selbständigen Lernen und sein Selbstvertrauen zu stärken. Die Schwächen der einen Praktik können teilweise durch Anwendung der anderen Praktik kompensiert werden. So kann es beispielsweise weniger frustrierend sein, wenn TDD als Paar gelernt wird. Allerdings gilt für beide Praktiken, dass die erfolgreiche Anwendung von der Persönlichkeit der beteiligten Entwickler abhängig ist. Wenig Probleme sollte die Kombination beider Praktiken machen, wenn der Berufseinsteiger bereits Grundkenntnisse in der testgetriebenen Entwicklung hat und über die für das PP nötigen Soft Skills verfügt.


% Vorteile (weniger Stress) für Entwickler belegbar?
% Es kann helfen, ist jedoch Abhängig. Aber welche Faktoren führen zum Erfolg von TDD, welche zum Scheitern? (Entwickler, Werkzeuge, Aufgabe)
% Efahrung, fähigkeiten IDE zu nutzen bzgl. TDDD? / Wenig Informationen über die Ausgangssituation (wenig Teilnehmer)
% Sicherstellen, dass TDD korrekt angewendet wurde? / Kaum Kontrolle der TDD-Druchführung
% Softwarequalität (langfristig) messen? / Aussagekraft der Kriterien zu Auswertung

% Produktivität, interne und externe Codeqaulität nur häufig verwendet, aber nicht die einzigen messbaren Werte. Außerdem verschiedenen Arten z.B. Code-Coverage zu messen Method-Coverage, Statement-Coverage und Conditional-Coverage

% Aufgaben zu kurz; für welche aufgaben ist TDD und PP geeigent?

% geringe Teilnehmer anzhal.
