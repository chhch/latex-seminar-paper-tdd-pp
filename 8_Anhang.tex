%\newpage
%\begin{table}[ht]
%\renewcommand{\arraystretch}{1.0} %1.3
%\caption{Entfernen}
%\label{table:TDDVergleichLiteraturReviews}
%\centering
%\begin{tabular}{@{}lllll@{}}
%\toprule
%Studie                                      & \cite{Munir2014ConsideringReview} & \cite{Bissi2016TheReview} & \cite{Rafique2013TheMeta-Analysis} & \cite{Kollanus2010Test-DrivenApproach}* \\ \midrule
%\cite{Bannerman2011AQuality}                & 60                                &                           &                                    &                                         \\
%\cite{Madeyski2010TheExperiment}            & 50                                & 42                        & 23                                 & 39                                      \\
%\cite{Crispin2006DrivingQuality}            & 65                                &                           &                                    &                                         \\
%\cite{Madeyski2005PreliminaryQuality}       & 40                                &                           & 48                                 & 23                                      \\
%\cite{Gupta2007AnDevelopment}               & 42                                & 23                        & 47                                 & 19                                      \\
%\cite{Muller2002ExperimentProgramming}      & 39                                &                           & 54                                 & 22                                      \\
%\cite{Flohr2006LessonsTeachings}            & 56                                &                           & 51                                 & 40                                      \\
%\cite{Desai2008AAcademia}                   & 73                                &                           &                                    &                                         \\
%\cite{Kollanus2008Test-drivenViewpoints}    & 13                                &                           &                                    &                                         \\
%\cite{Flohr2005AnProblems}                  & 52                                &                           &                                    &                                         \\
%\cite{Pancur2011ImpactExperiment}           & 41                                & 10                        & 49                                 &                                         \\
%\cite{Siniaalto2007ACoverage}               & 61                                & 20                        & 20                                 & 42                                      \\
%\cite{Geras2004ADevelopment}                & 49                                & 21                        & 35                                 & 44                                      \\
%\cite{George2004ADevelopment}               & 46                                & 22                        & 33                                 & 18                                      \\
%\cite{George2003AnIndustry}                 & 82                                & 24                        & 27                                 & 47                                      \\
%\cite{Maximilien2003AssessingIBM}           & 69                                & 25                        & 29                                 & 24                                      \\
%\cite{Janzen2008DoesQuality}                & 44                                & 26                        & 22                                 & 10                                      \\
%\cite{Vu2009EvaluatingProject}              & 45                                & 27                        & 57                                 & 46                                      \\
%\cite{Pancur2003TowardsEnvironment}         & 3                                 & 28                        & 55                                 & 21                                      \\
%\cite{Janzen2007EmpiricalCourses}           &                                   & 29                        & 19                                 & 45                                      \\
%\cite{Edwards2004UsingReflection-in-action} &                                   & 30                        & 31                                 & 7                                       \\
%\cite{Janzen2008Test-drivenCourses}         & 51                                & 31                        &                                    &                                         \\
%\cite{Canfora2006EvaluatingProfessionals}   & 43                                & 32                        & 40                                 & 37                                      \\
%\cite{Bhat2006EvaluatingStudies}            & 64                                & 33                        & 37                                 & 9                                       \\
%\cite{Turnu2006ModelingPractice}            &                                   & 34                        &                                    &                                         \\
%\cite{Janzen2006OnDesign}                   & 54                                & 35                        & 41                                 & 6                                       \\
%\cite{Erdogmus2005OnProgramming}            & 6                                 & 36                        & 5                                  & 5                                       \\
%\cite{Sanchez2007OnIBM}                     & 74                                & 37                        & 16                                 & 26                                      \\
%\cite{Aniche2012HowFeedback}                &                                   & 38                        &                                    &                                         \\
%\cite{Slyngstad2008TheStudy}                & 63                                & 39                        & 63                                 & 25                                      \\
%\cite{Aniche2010MostDevelopers}             & 72                                & 40                        &                                    &                                         \\
%\cite{Damm2006ResultsDevelopment}           & 62                                & 41                        & 25                                 &                                         \\
%\cite{Madeyski2007TheStudy}                 & 53                                & 43                        & 26                                 & 31                                      \\
%\cite{Nagappan2008RealizingTeams}           & 66                                & 44                        & 38                                 & 8                                       \\
%\cite{Yenduri2006ImpactStudy.}              & 87                                & 45                        & 59                                 &                                         \\
%\cite{Williams2003Test-drivenPractice}      & 67                                &                           & 30                                 & 27                                      \\
%\cite{Edwards2003UsingPerformance}          & 81                                &                           & 32                                 & 32                                      \\
%\cite{Desai2009ImplicationsCurricula}       & 48                                &                           & 50                                 & 34                                      \\
%\cite{Huang2009EmpiricalProgramming}        & 5                                 &                           & 52                                 & 20                                      \\
%\cite{Kaufmann2003ImplicationsStudy}        & 88                                &                           & 53                                 &                                         \\
%\cite{Rahman2007ApplyingCourses}            & 80                                &                           & 56                                 & 33                                      \\
%\cite{Xu2009EvaluationStudy}                & 83                                &                           & 58                                 & 35                                      \\
%\cite{Zhang2006ComparisonProject}           & 79                                &                           & 60                                 & 48                                      \\
%\cite{Dogsa2011TheStudy}                    & 59                                &                           & 61                                 &                                         \\
%\cite{George2002AnalysisApproach}           &                                   &                           & 28                                 &                                         \\
%\cite{Janzen2006AnQuality}                  &                                   &                           & 42                                 &                                         \\
%\cite{Lui2004TestChina}                     &                                   &                           & 62                                 & 30                                      \\
%\cite{Melnik2005AMethods}                   & 84                                &                           & 10                                 &                                         \\
%\cite{Rendell2008EffectiveDevelopment}      &                                   &                           & 12                                 & 41                                      \\
%\cite{Abrahamsson2005ImprovingDevelopment}  & 85                                &                           & 15                                 &                                         \\
%\cite{Siniaalto2008DoesStudy}               & 71                                &                           & 21                                 & 43                                      \\
%\cite{Ynchausti2001IntegratingProcess}      & 86                                &                           & 24                                 & 29                                      \\
%\cite{Kollanus2008UnderstandingExperiments} &                                   &                           & 11                                 & 53                                      \\
%\cite{Langr2001EvolutionDesign}             &                                   &                           & 13                                 &                                         \\
%\cite{Steinberg2001TheCourse}               &                                   &                           & 14                                 & 12                                      \\
%\cite{Madeyski2006TheExperiment}            &                                   &                           & 17                                 & 38                                      \\
%\cite{Siniaalto2006TheQuality}              &                                   &                           & 18                                 &                                         \\
%\cite{Geras2004TheDevelopment}              &                                   &                           & 34                                 &                                         \\
%\cite{Hanhineva2004Test-drivenEnvironment}  &                                   &                           & 36                                 &                                         \\
%\cite{Canfora2006ProductivityProfessionals} &                                   &                           & 39                                 & 36                                      \\
%\cite{Damm2007QualityDevelopment}           &                                   &                           & 71                                 & 28                                      \\ \bottomrule
%\end{tabular}
%\end{table}

\begin{table*}[t]
\renewcommand{\arraystretch}{1.3}
\caption{Vorabuntersuchung}
\label{table:TDDVorabuntersuchung}
\centering
\begin{tabularx}{\textwidth}{@{}Xp{0.29\textwidth}p{0.24\textwidth}p{0.07\textwidth}p{0.28\textwidth}@{}}
\toprule
Studie                                       & Beschreibung                                                                                                                                                                                                                             & Teilnehmer                                                                                                                                                     & Umfeld                    & Ergebnis                                                                                                                                                                                                                                                        \\ \midrule
\cite{Canfora2006ProductivityProfessionals}  & Vergleich TDD mit TDA ("Test After Coding") hinsichltich der Produktivität.                                                                                                                                                              & 28 Mitarbeiter eines Unternehmens mit einem BsC in Computer Science.                                                                                           & industriell               & Die Arbeit mit TDD war signifikant unproduktiver.                                                                                                                                                                                                               \\
\cite{Dogsa2011TheStudy}                     & Drei mittelgroße Projekt wurden beobachtet. Ein Projekt wurde mit TDD entwickelt, die anderen beiden ohne.                                                                                                                               & Die 36 Teilnehmer hatten im Durchschnitt mehr als 5 Jahre Erfahrung als professionelle Entwickler.                                                             & industriell               & Der Code des TDD-Projekts hatte eine höhere Qualität und war einfacher zu warten. Die Produktivität war geringer.                                                                                                                                               \\
\cite{Erdogmus2005OnProgramming}             & Studenten wurden in zwei Gruppen eingeteilt. Eine Gruppe entwickelte nach TDD, die andere nach TLD.                                                                                                                                      & Computer Science Studenten ohne einen Abschluss.                                                                                                               & akademisch                & Die Gruppe nicht nach TDD entwickelte war produktiver.                                                                                                                                                                                                          \\
\cite{Fucci2016ATest-Last}                   & TDD-Entwickler wurden bei ihrer Arbeit beobachtet.                                                                                                                                                                                       & 39 professionelle Entwickler                                                                                                                                   & industriell               & Vorteil von TDD ensteht durch die Fokussierung auf kleinschrittiges Vorgehen.                                                                                                                                                                                   \\
\cite{Fucci2016AnApproach}                   & Studenten haben in TDD und TLD entwickelt.                                                                                                                                                                                               & 21 Studenten mit einem Abschluss                                                                                                                               & akademisch                & Kein signifikanter Unterschied zwischen TDD und TLD in Testaufwand, externe Codequalität und Produktivität                                                                                                                                                      \\
\cite{Fucci2015TowardsStudy}                 & Experiment mit verschiedenen Gruppen, die nach TDD-Fähigkeiten der Entwickler eingeteilt wurden.                                                                                                                                         & 30 professionelle Entwickler                                                                                                                                   & industriell               & Keine signifikanten Unterschiede zwischen den Gruppen, hinsichtlich externen Codequalität und Produktivität.                                                                                                                                                    \\
\cite{Hilton2009QuantitativelyProjects}      & Der Code von Open Source Projekten wurde untersucht.                                                                                                                                                                                     & 4 TDD Projekte und 4 nicht-TDD Projekte.                                                                                                                       & -                         & Die Projekte, die nach TDD vorgingen, konnte eine bessere interne Codequalität vorweisen.                                                                                                                                                                       \\
\cite{Janzen2006OnDesign}                    & Studenten wurden in drei Gruppen eingeteilt. Eine Gruppe entwickelte nach TDD, die andere nach iterativen TLD und eine weitere nach wasserfall TLD.                                                                                      & Studenten ohne Abschluss, die mindestens zwei Programmierkurse absolviert hatten.                                                                              & akademisch                & Die Projekte, die nach TDD vorgingen, konnten eine bessere interne Codeqaulität, externe Codequalität und Produktivität vorweisen.                                                                                                                              \\
\cite{Muller2007TheProcess}                  & Der TDD-Entwicklungsprozess von anfängern und experten wurde untersucht.                                                                                                                                                                 & Zu den Teilnehmern gehörten 11 Computer Science Studenten und 7 professionellen Entwickler, die bereits Erfahrung in TDD hatten.                               & akademisch und insutriell & Die Experten hatten sich mehr an die Regeln des TDD gehalten und ihre Tests hatten eine höhere Qualität.                                                                                                                                                        \\
\cite{Nagappan2008RealizingTeams}            & Fallstudien wurden mit drei Teams von Microsoft und einem Team von IBM durchgeführt. Alle Teams entwickelten testgetrieben.                                                                                                              & 27-30 professionelle Entwickler, mit jeweils zwischen 6 und 10 Jahren erfahrung.                                                                               & industriell               & Die Fehlerdichte der vier Produkte nahm zwischen 40\% und 90\% ab, relativ zu ähnlichen Projekte, welche kein TDD einsetzten.                                                                                                                                   \\
\cite{Nanthaamornphong2015Test-DrivenSurvey} & Befragung von Entwickler die im akademischen Bereich arbeiten.                                                                                                                                                                           & 77 Personen haben an der Umfrage teilgenommen, davon hatte 64 erfahrung mit TDD.                                                                               & akademisch                & TDD hat ein positiven Effekt auf die Qualität wissenschaftlicher Software, benötigt aber viel Aufwand.                                                                                                                                                          \\
\cite{Parodi2016ComparingProgramming}        & Der Quellcode von Studenten wurde mit einem Code-Analyser untersucht. Die Studenten entwickelten entweder nach TDD, TLD oder ad hoc.                                                                                                     & 75 Studenten ohne Abschluss.                                                                                                                                   & akademisch                & Es wurden zwischen den verschiedenen Entwicklungspraktiken, keine signifikanten Unterschiede hinsichtlich der technischer Schulden gefunden.                                                                                                                    \\
\cite{Sanchez2007OnIBM}                      & Analyse eines Teams von IBM, welches seit 5 Jahren testgetrieben entwickelt.                                                                                                                                                             & Alle Mitarbeiter hatten mindestens einen Bachelor-Abschluss. Das Team bestand aus 7 bis 17 Personen.                                                           & industriell               & Die Qualität der Software verbesserte sich. Die Produktivität und Komplexität wurde geringer.                                                                                                                                                                   \\
\cite{Siniaalto2008DoesStudy}                & Fallstudie mit fünf kleinen Softwareprojekten. Bei drei wurde nach TDD und bei zwei nach TLD gearbeitet.                                                                                                                                 & 4 der 5 Teams bestanden aus Studenten, die einen Abschluss hatten. Ein Team bestand aus professionellen Entwicklern. Die Teams bestanden aus 2 bis 5 Personen. & akademisch                & Die Unterschiede im Programmcode waren nicht so deutlich, wie erwartet.                                                                                                                                                                                         \\
\cite{Siniaalto2007ACoverage}                & Untersuchung der Effekte von TDD anhand von drei Softwareprojekten.                                                                                                                                                                      & Teams bestanden aus 4 oder 5 Personen.                                                                                                                         & industriell               & Der Einfluss von TDD auf das Programmdesign war nicht so evident wie erwartet, aber die Testabdeckung war signifikant höher als beim TLD.                                                                                                                       \\
\cite{Vu2009EvaluatingProject}               & Ergebnisse von Studenten die an einem einjährigen Software-Engineering Kurs teilnahmen wurden untersucht. Die Studenten wurden in zwei TDD-Gruppen und eine TDL-Gruppe eingeteilt.                                                       & 40 Stundenten ohne Abschluss                                                                                                                                   & akademisch und insutriell & Die TLD-Gruppe war produktiver und hat mehr Tests geschrieben.                                                                                                                                                                                                  \\
\cite{Wilkerson2012ComparingDevelopment}     & Studenten wurden in vier Gruppen eingeteilt und mussten die selbe Programmieraufgabe lösen. Eine Gruppe entwickelte testgetrieben. Eine Gruppe entwickelte nicht testgetrieben. (Die anderen beiden Gruppen verwendete Code Inspection.) & Insgesamt 29 Studenten ohne Abschluss. Die TDD-Gruppe bestand aus 9 Studenten. Die Kontroll-Gruppe aus 7.                                                      & akademisch                & TDD war bei der Reduzierung von Fehlern nicht effektiver als traditionellen Programmierpraktiken.                                                                                                                                                               \\ \bottomrule
\end{tabularx}
\end{table*}

\addtocounter{table}{-1}
\begin{table*}[p]
% increase table row spacing, adjust to taste
\renewcommand{\arraystretch}{1.3}
% if using array.sty, it might be a good idea to tweak the value of
% \extrarowheight as needed to properly center the text within the cells
\caption{Vorabuntersuchung (Fortsetzung)}
%\label{table:Vorabuntersuchung_fort}
\centering
% Some packages, such as MDW tools, offer better commands for making tables
% than the plain LaTeX2e tabular which is used here.
\begin{tabularx}{\textwidth}{@{}Xp{0.29\textwidth}p{0.24\textwidth}p{0.07\textwidth}p{0.28\textwidth}@{}}
\toprule
Studie                                       & Beschreibung                                                                                                                                                                                                                             & Teilnehmer                                                                                                                                                     & Umfeld                    & Ergebnis                                                                                                                                                                                                                                                        \\ \midrule
\cite{Xu2007ProgrammersDevelopment}          & Fortgeschrittene und Experten sollten in Paaren Programmieraufgaben mit TDD lösen. Der Prozess wurde auf Video aufgezeichnet und analysiert.                                                                                             & 16 Studenten mit Abschluss und 4 Studenten ohne Abschluss. 8 fortgeschrittene Paare und ein Paar aus Experten.                                                     &                           & Experten vielen die Aufgabe einfacher.                                                                                                                                                                                                                          \\
\cite{Xu2009EvaluationStudy}                 & Studenten haben eine einfache Anwendung entweder mit TDD oder ohne entwickelt.                                                                                                                                                           & 8 Studenten ohne Abschluss. Keine Erfahrung in TDD. Ähnliche Programmierfähigkeiten.                                                                           & akademisch                & Der Code der Studenten, die testgestrieben entwickelten, hatte eine höhere interne und externe Codequalität. Außerdem waren die Studenten produktiver.                                                                                                          \\
\cite{Yahya2015TheDevelopment}               & Die Studie vergleicht TDD mit TLD und untersucht, welchen Einfluss die Programmierkompetenz auf die Qualität hat.                                                                                                                        & Studenten.                                                                                                                                                     & akademisch                & Die Kopplung war, unabhängig von der Kompetenz, bei TDD immer geringer als bei TLD. Bei Hohe und mittlere Programmierkompetenzen führten die Entwicklung nach TDD zu weniger komplexen Code. TLD führte bei geringer und mittlere Kompetenz zu mehr Codezeilen. \\
\cite{Causevic2013EffectsExperiment} & Die Entwickler eines Unternehmens wurden in drei Gruppen eingeteilt: TDD, TDD mit negativen Tests und TLD ("Test Last Development"). Der produzierte Quellcode und die Testfälle wurden hinsichtlich ihrer Qualität untersucht. & 33 professionelle Entwickler                                                                              & industriell               &                                                                                                                                                            \\
\cite{Marchenko2009Long-TermStudy}   & Untersucht die Effekte von TDD, indem es ein Team interviewt, welches TDD (mit PP) 3 Jahre angewendet hat.                                                                                                                      & 8 Mitarbeiter                                                                                             & industriell               & Durch TDD hat das Vertrauen des Teams in die Codequalität und die Wartbarkeit zugenommen. Das Team hat keine signifikanten negativen Effekte wahrgenommen. \\
\cite{Romano2017FindingsDevelopment} & Die Teilnehmer der Studie sollten ein neues Feature für eine bereits existierenden Anwendung mit TDD und Pair Programming implementieren.                                                                                       & 14 Computer Science Studenten mit Abschluss. 6 professionelle Entwickler mit mehr als 10 Jahre erfahrung. & akademisch und insutriell & Die Entwickler schreiben quick-and-dirty Code der die Tests erfüllt, aktualisieren ihre Tests selten und ignorieren den Refactor-Schritt.                  \\
\cite{Xu2006EmpiricalDevelopment}    & Studenten haben eine einfache Spieleanwendung entwickelt. Eine Gruppe hat XP-Praktiken, wie Pair Programming und TDD genutzt. Die andere Gruppe hat nach dem Wasserfall-Modell entwickelt.                                      & 12 Studenten ohne Abschluss. 8 Studenten entwickelten in Paaren mit TDD.                                  & akademisch                & Die Studenten, die in Pair Programming entwickelt haben, waren produktiver und die Codequalität war besser.                                                \\ \bottomrule
\end{tabularx}
\end{table*}

\begin{table*}[ht]
\renewcommand{\arraystretch}{1.3}
\caption{Auswertung TDD}
\label{table:TDDAuswertungKomplett}
\centering
\begin{threeparttable}
\begin{tabularx}{\textwidth}{@{}lXXXXXXll@{}}
\toprule
Studie/LR                                & Munir \cite{Munir2014ConsideringReview} & Bissi \cite{Bissi2016TheReview} & Kollanus \cite{Kollanus2010Test-DrivenApproach} & Jeffries \cite{Jeffries2007TheProgramming} & Sfetsos \cite{Sfetsos2010EmpiricalReview} & Causevic \cite{Causevic2011FactorsReview} & Suche                & U                                                 \\ \midrule
\cite{Madeyski2010TheExperiment}         & n\textsuperscript{=} i\textsuperscript{=} e\textsuperscript{=}                    & i\textsuperscript{=}                          & i\textsuperscript{=}                                          &                                            &                                           & i\textsuperscript{=}                                    & G                    & n\textsuperscript{=} i\textsuperscript{=} e\textsuperscript{=}                              \\
\cite{Madeyski2005PreliminaryQuality}    & e\textsuperscript{=}                                  &                                 & e\textsuperscript{-}                                          & e\textsuperscript{-}                                     &                                           &                                           &                      & e\textsuperscript{-/=}                                          \\
\cite{Gupta2007AnDevelopment}            & t\textsuperscript{=} e\textsuperscript{+} p\textsuperscript{=}                    & e\textsuperscript{=} p\textsuperscript{+}                   & e\textsuperscript{+} p\textsuperscript{+}                                   &                                            & t\textsuperscript{-} e\textsuperscript{+} p\textsuperscript{+}                      & t\textsuperscript{+}                                    &                      & t\textsuperscript{-/=/+} e\textsuperscript{=/+} p\textsuperscript{=/+}                      \\
\cite{Muller2002ExperimentProgramming}   & c\textsuperscript{+} t\textsuperscript{+} i\textsuperscript{=} e\textsuperscript{=}             &                                 & e\textsuperscript{=} p\textsuperscript{=}                                   & i\textsuperscript{=} p\textsuperscript{=}                              & e\textsuperscript{=}                                    & t\textsuperscript{-}                                    &                      & c\textsuperscript{+} t\textsuperscript{-/+} i\textsuperscript{=} e\textsuperscript{=} p\textsuperscript{=}              \\
\cite{Causevic2012TestExperiment}        &                                         &                                 &                                                 &                                            &                                           &                                           & i\textsuperscript{=} e\textsuperscript{+} p\textsuperscript{+} &                                                   \\
\cite{Fucci2016AnApproach}               &                                         &                                 &                                                 &                                            &                                           &                                           & e\textsuperscript{=} p\textsuperscript{=}        &                                                   \\ \bottomrule
\end{tabularx}
\medskip
      \footnotesize\textbf{Legende:}\smallskip
      \begin{tablenotes}\footnotesize
      \item[p] Produktivität
      \item[e] externe Qualität
      \item[i] interne Qualität
      \item[t] effort/time
      \item[s] code size
      \item[n] Nummer Tests
      \item[c] conformance
      \item[+/=/-] positiv/neutral/negativ
      \end{tablenotes}
\end{threeparttable}
\end{table*}
